% !TeX root = ../../main.tex
\graphicspath{{Parts/1_Introductions/graphics/}}

\LaTeX{}{} is a high quality typesetting environment designed for creating scientific and technical documents. It is the standard for any of our publications or official scientific communication. Perhaps the easiest and most common way to use \LaTeX{}{} is through \hyperlink{https://www.overleaf.com}{Overleaf}, an online \LaTeX{} environment that includes a bunch of useful features such as the visual editor for first-time \LaTeX{} users, and real-time collaboration.

\subsection{So why use a local version if Overleaf is this good?}
Obviously, Overleaf is amazing and I do not intend to disparage its usage here. But one of the biggest flaws with using Overleaf exclusively is the fact that it is an \emph{online} editor. This means that you need to be constantly connected to the internet to use Overleaf to edit your documents. This means that in cases like power/server outages, or times when you're travelling without an internet connection (say a long flight on the way to a conference), you can't use Overleaf. In such cases, it is good to have a local option to work in \LaTeX{} offline. Additionally, depending on the code editor you use to write your local \LaTeX{} documents, you may have some nice integrations/shortcuts to exploit.

\subsection{The purpose and scope of this writeup}
This writeup is mainly intended for \LaTeX{} users who are comfortable with using Overleaf for preparing documents. As such, this document will assume basic understanding of the \LaTeX{} environment. The remainder of this writeup is split into two sections, namely
\begin{itemize}
    \item \textbf{Installation and Setup}: Talking about how to install and use \LaTeX{} locally, and providing some personalization/workflow tips
    \item \textbf{Overleaf Integration}: Exploring integrating local \LaTeX{} with Overleaf
\end{itemize}

\subsection{More resources}
\begin{itemize}
    \item 
    The source code for this writeup and the examples shown can be found in the public Github repo accessible at \url{https://github.com/kaushikr64/Local-LaTeX-guide}
    
    \item 
    Quite a bit of information on the VSCode frontend for \LaTeX{} can be found at the site for it \url{https://github.com/James-Yu/LaTeX-Workshop/wiki}
\end{itemize}





%\clearpage
%\subsubsection{Code Listing}
%\lstinputlisting[style=Matlab-editor,caption={Problem H1 - Finding a Poincaré section (same code for all cases)}]{../Code/H1/main05.m}



    